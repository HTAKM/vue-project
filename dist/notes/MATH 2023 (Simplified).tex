\documentclass{huhtakm-template-book}
\usepackage{tikz-cd}
\newtheorem{steps}{Steps}[definition]
\newenvironment{step}[0]{\begin{steps}\begin{mdframed}[backgroundcolor=green!5]}{\end{mdframed}\end{steps}}
\DeclareMathOperator{\proj}{proj}
\DeclareMathOperator{\comp}{comp}
\DeclareMathOperator{\Vol}{Vol}
\DeclareMathOperator{\N}{N}
\DeclareMathOperator{\J}{J}
\title{MATH 2023: Multivariable Calculus \\ (Summary)}
\author{
	HU-HTAKM\\
	Website: \url{https://htakm.github.io/htakm_test/}
}
\date{
	Last major change: August 30, 2024\\
	Last small update (fixed typo): September 12, 2025
}
\begin{document}
\maketitle
\chapter{Three-Dimensional Space}
\begin{defn}(Vector Additions and Scalar Multiplications)
    Let $\mathbf{a}=\left<a_{1},a_{2},a_{3}\right>$ and $\mathbf{b}=\left<b_{1},b_{2},b_{3}\right>$ be two vectors in $\mathbb{R}^{3}$, and $c$ be a scalar. Then
    \begin{align*}
        \tag{\textbf{Addition}}
        \mathbf{a}+\mathbf{b}&=\left<a_{1}+b_{1},a_{2}+b_{2},a_{3}+b_{3}\right>\\
        \tag{\textbf{Scalar multiplication}}
        c\mathbf{a}&=\left<ca_{1},ca_{2},ca_{3}\right>
    \end{align*}
    \textbf{Negative} of a vector is $-\mathbf{a}=(-1)\mathbf{a}$.\\
    \textbf{Difference} between vectors is $\mathbf{a}-\mathbf{b}=\mathbf{a}+(-\mathbf{b})$.
\end{defn}
\begin{lem}
    The following properties hold:
    \begin{enumerate}
        \item Commutative rule: $\mathbf{a}+\mathbf{b}=\mathbf{b}+\mathbf{a}$
        \item Associative rule: $(\mathbf{a}+\mathbf{b})+\mathbf{c}=\mathbf{a}+(\mathbf{b}+\mathbf{c})$
        \item Distributive rule: $(\lambda+\mu)\mathbf{a}=\lambda\mathbf{a}+\mu\mathbf{a}$ and $\lambda(\mathbf{a}+\mathbf{b})=\lambda\mathbf{a}+\lambda\mathbf{b}$
    \end{enumerate}
\end{lem}
\begin{defn}(\textbf{Dot product})
    Let $\mathbf{a}=\left<a_{1},a_{2},a_{3}\right>$ and $\mathbf{b}=\left<b_{1},b_{2},b_{3}\right>$. The dot product between vectors $\mathbf{a}$ and $\mathbf{b}$ is defined as:
    \begin{equation*}
        \mathbf{a}\cdot\mathbf{b}=a_{1}b_{1}+a_{2}b_{2}+a_{3}b_{3}
    \end{equation*}
\end{defn}
\begin{defn}(\textbf{Length})
    Let $\mathbf{a}=\left<a_{1},a_{2},a_{3}\right>$. The length of vector $\mathbf{a}$ is given by:
    \begin{equation*}
        \abs{\mathbf{a}}=\sqrt{\mathbf{a}\cdot\mathbf{a}}=\sqrt{a_{1}^{2}+a_{2}^{2}+a_{3}^{2}}
    \end{equation*}
\end{defn}
\begin{lem}
    The following properties hold:
    \begin{enumerate}
        \item $\mathbf{a}\cdot\mathbf{b}=\mathbf{b}\cdot\mathbf{a}$
        \item $(\mathbf{a}+\mathbf{b})\cdot\mathbf{c}=\mathbf{a}\cdot\mathbf{c}+\mathbf{b}\cdot\mathbf{c}$
        \item $(\lambda\mathbf{a})\cdot\mathbf{b}=\lambda(\mathbf{a}\cdot\mathbf{b})$
        \item $\mathbf{0}\cdot\mathbf{a}=\mathbf{a}\cdot\mathbf{0}=0$
    \end{enumerate}
\end{lem}
\begin{thm}
    Let $\mathbf{a}=\left<a_{1},a_{2},a_{3}\right>$ and $\mathbf{b}=\left<b_{1},b_{2},b_{3}\right>$ be two vectors in $\mathbb{R}^{3}$, and $\theta$ be the angle between them. Then:
    \begin{equation*}
        \mathbf{a}\cdot\mathbf{b}=\abs{\mathbf{a}}\abs{\mathbf{b}}\cos\theta
    \end{equation*}
\end{thm}
\begin{cor}
    Two non-zero vectors $\mathbf{a}$ and $\mathbf{b}$ are orthogonal if and only if $\mathbf{a}\cdot\mathbf{b}=0$.
\end{cor}

\newpage
\begin{defn}(\textbf{Projection})
    Let $\mathbf{a}$ and $\mathbf{b}$ be two vectors in $\mathbf{R}^{3}$. The \textbf{scalar projection} of $\mathbf{b}$ onto $\mathbf{a}$ is the signed length:
    \begin{equation*}
        \comp_{\mathbf{a}}(\mathbf{b})=\frac{\mathbf{a}\cdot\mathbf{b}}{\abs{\mathbf{a}}}
    \end{equation*}
    The \textbf{vector projection} of $\mathbf{b}$ onto $\mathbf{a}$ is a vector:
    \begin{equation*}
        \proj_{\mathbf{a}}(\mathbf{b})=\comp_{\mathbf{a}}(\mathbf{b})\frac{\mathbf{a}}{\abs{\mathbf{a}}}=\frac{\mathbf{a}\cdot\mathbf{b}}{\abs{\mathbf{a}}^{2}}\mathbf{a}
    \end{equation*}
\end{defn}
\begin{defn}(\textbf{Cross product})
    Let $\mathbf{a}=a_{1}\mathbf{i}+a_{2}\mathbf{j}+a_{3}\mathbf{k}$ and $\mathbf{b}=b_{1}\mathbf{i}+b_{2}\mathbf{j}+b_{3}\mathbf{k}$ be vectors in $\mathbb{R}^{3}$ with angle $\theta$ between them. The cross product $\mathbf{a}\times\mathbf{b}$ between $\mathbf{a}$ and $\mathbf{b}$ is defined as a vector such that
    \begin{enumerate}
        \item Length $\abs{\mathbf{a}\times\mathbf{b}}=\abs{\mathbf{a}}\abs{\mathbf{b}}\sin\theta$
        \item The cross product $\mathbf{a}\times\mathbf{b}$ is orthogonal to both $\mathbf{a}$ and $\mathbf{b}$
        \item Direction is determined by the right-hand grab rule
    \end{enumerate}
\end{defn}
\begin{lem}
    The following properties hold:
    \begin{enumerate}
        \item $\mathbf{a}\times\mathbf{b}=-\mathbf{b}\times\mathbf{a}$
        \item $(\mathbf{a}+\mathbf{b})\times\mathbf{c}=\mathbf{a}\times\mathbf{c}+\mathbf{b}\times\mathbf{c}$
        \item $\mathbf{a}\times\mathbf{0}=\mathbf{0}$
        \item $\mathbf{a}\times\mathbf{a}=\mathbf{0}$
    \end{enumerate}
\end{lem}
\begin{thm}(Determinant Formula of cross product)
    Let $\mathbf{a}=a_{1}\mathbf{i}+a_{2}\mathbf{j}+a_{3}\mathbf{k}$ and $\mathbf{b}=b_{1}\mathbf{i}+b_{2}\mathbf{j}+b_{3}\mathbf{k}$. Their cross product is given by
    \begin{equation*}
        \mathbf{a}\times\mathbf{b}=\begin{vmatrix}
            \mathbf{i} & \mathbf{j} & \mathbf{k}\\
            a_{1} & a_{2} & a_{3}\\
            b_{1} & b_{2} & b_{3}
        \end{vmatrix}=(a_{2}b_{3}-a_{3}b_{2})\mathbf{i}-(a_{1}b_{3}-a_{3}b_{1})\mathbf{j}+(a_{1}b_{2}-a_{2}b_{1})\mathbf{k}
    \end{equation*}
\end{thm}
\begin{lem}
    The following properties hold:
    \begin{enumerate}
        \item Area of the parallelogram formed by $\mathbf{a}$ and $\mathbf{b}$ is $\abs{\mathbf{a}\times\mathbf{b}}$.
        \item Area of the triangle formed by $\mathbf{a}$ and $\mathbf{b}$ is $\frac{1}{2}\abs{\mathbf{a}\times\mathbf{b}}$
        \item $\mathbf{a}\times\mathbf{b}=\mathbf{0}$ if and only if $\mathbf{a}$ and $\mathbf{b}$ are parallel.
        \item Volume of the parallelepiped spanned by $\mathbf{a}$, $\mathbf{b}$, and $\mathbf{c}$ is $\abs{\mathbf{a}\cdot(\mathbf{b}\times\mathbf{c})}$.
    \end{enumerate}
\end{lem}
\begin{defn}(Parametric equation of a line)
    Suppose a line $L$ passes through the point $P_{0}(x_{0},y_{0},z_{0})$ and is parallel to a vector $\mathbf{v}=\left<v_{1},v_{2},v_{3}\right>$. The parametric equation is given by
    \begin{equation*}
        \begin{cases}
            x=x_{0}+tv_{1}\\
            y=y_{0}+tv_{2}\\
            z=z_{0}+tv_{3}
        \end{cases}
    \end{equation*}
    In vector form, it is given by
    \begin{equation*}
        \mathbf{r}(t)=\left<x_{0}+tv_{1},y_{0}+tv_{2},z_{0}+tv_{3}\right>
    \end{equation*}
\end{defn}
\begin{thm}
    Given two lines in $\mathbb{R}^{3}$. There are 4 possible relative positions.
    \begin{enumerate}
        \item Same (Same direction and have common points)
        \item Parallel (Same direction but no common points)
        \item Skew (Different direction and no common points)
        \item Intersect (Different direction but have common point)
    \end{enumerate}
\end{thm}
\begin{defn}
    Given a plane $P$ with a normal vector $\mathbf{n}=\left<a,b,c\right>$. Assume that it passes through $P_{0}(x_{0},y_{0},z_{0})$. Then the equation of the plane is given by
    \begin{equation*}
        ax+by+cz=ax_{0}+by_{0}+cz_{0}
    \end{equation*}
\end{defn}
\begin{thm}
    Given two planes in $\mathbb{R}^{3}$. There are 3 possible relative positions.
    \begin{enumerate}
        \item Same (Same normal vector and have common points)
        \item Parallel (Same normal vector but no common points)
        \item Intersect (Different normal vector)
    \end{enumerate}
\end{thm}
\begin{defn}(Parametric equation of a curve)
    A parametric equation of a curve is of the form
    \begin{align*}
        &\begin{cases}
            x=f(t)\\
            y=g(t)\\
            z=h(t)
        \end{cases} & \mathbf{r}(t)&=f(t)\mathbf{i}+g(t)\mathbf{j}+h(t)\mathbf{k}
    \end{align*}
\end{defn}
\begin{defn}(Derivatives of parametric curves)
    The derivative of a parametric curve is given by
    \begin{equation*}
        \mathbf{r}'(t)=f'(t)\mathbf{i}+g'(t)\mathbf{j}+h'(t)\mathbf{k}
    \end{equation*}
\end{defn}
\begin{lem}
    The following properties hold:
    \begin{enumerate}
        \item $\odv{}{t}(f(t)\mathbf{u}(t))=f'(t)\mathbf{u}(t)+f(t)\mathbf{u}'(t))$
        \item $\odv{}{t}(\mathbf{u}(t)\cdot\mathbf{v}(t))=\mathbf{u}'(t)\cdot\mathbf{v}(t)+\mathbf{u}(t)\cdot\mathbf{v}'(t)$
        \item $\odv{}{t}(\mathbf{u}(t)\times\mathbf{v}(t))=\mathbf{u}'(t)\times\mathbf{v}(t)+\mathbf{u}(t)\times\mathbf{v}'(t)$
    \end{enumerate}
\end{lem}

\chapter{Partial Differentiations}
\begin{defn}(\textbf{Limit})
    Given a function $f(x,y)$ and a point $P_{0}(x_{0},y_{0})$. We say that $\lim_{(x.y)\to(x_{0},y_{0}})f(x,y)=L$ of for all $\varepsilon>0$, there exists $\delta>0$ such that
    \begin{equation*}
        \abs{f(x,y)-f(x_{0},y_{0})}<\varepsilon
    \end{equation*}
    for all $P(x,y)$ such that
    \begin{equation*}
        \sqrt{(x-x_{0})^{2}+(y-y_{0})^{2}}<\delta
    \end{equation*}
\end{defn}
\begin{thm}(Squeeze Theorem)
    Given functions $f,g,h$. If $f(x,y,z)\leq g(x,y,z)\leq h(x,y,z)$ and suppose that
    \begin{equation*}
        \lim_{(x,y,z)\to(a,b,c)}f(x,y,z)=\lim_{(x,y,z)\to(a,b,c)}h(x,y,z)=L
    \end{equation*}
    Then
    \begin{equation*}
        \lim_{(x,y,z)\to(a,b,c)}g(x,y,z)=L
    \end{equation*}
\end{thm}
\begin{defn}(\textbf{Continuity})
    Given a function $f(x,y)$. A function $f$ is continuous at $(x_{0},y_{0})$ if
    \begin{equation*}
        \lim_{(x,y)\to(x_{0},y_{0})}f(x,y)=f(x_{0},y_{0})
    \end{equation*}
    Function $f$ is continuous if it is continuous at all points in the domain. 
\end{defn}
\begin{defn}(\textbf{Level curve})
    Given a function $f(x,y)$.
    Fix $h\in\mathbb{R}$. The level curve of $f$ at $h$ is given by $f(x,y)=h$.\\
    Combining multiple level curves give us a \textbf{contour map}.
\end{defn}
\begin{defn}(\textbf{Partial derivatives})
    Given a function $f(x,y)$. The partial derivatives of $f(x,y)$ with respect to $x$ and $y$ are:
    \begin{align*}
        f_{x}(x,y)=\pdv*{f(x,y)}{x}&=\lim_{h\to 0}\frac{f(x+h,y)-f(x,y)}{h} & f_{y}(x,y)=\pdv*{f(x,y)}{y}&=\lim_{h\to 0}\frac{f(x,y+h)-f(x,y)}{h}
    \end{align*}
\end{defn}
\begin{defn}(\textbf{Second partial derivatives})
    The second partial derivatives are defined as follows:
    \begin{align*}
        f_{xx}&=\pdv[order={2}]{f}{x}=\pdv*{\pdv{f}{x}}{x} & f_{xy}&=\pdv{f}{y,x}=\pdv*{\pdv{f}{x}}{y} & f_{yx}&=\pdv{f}{x,y}=\pdv*{\pdv{f}{y}}{x} & f_{yy}&=\pdv[order={2}]{f}{y}=\pdv*{\pdv{f}{y}}{y}
    \end{align*}
\end{defn}
\begin{thm}(Mixed Partial Theorem)
    Given a function $f(x,y)$. If at least one of the second partials $f_{xy}$ and $f_{yx}$ exists and is continuous, then $f_{xy}=f_{yx}$.
\end{thm}
\begin{lem}(Chain rule)
    Given a function $f(x,y,z)$, where $x,y,z$ are functions of $t$. Then we have
    \begin{equation*}
        \odv{f}{t}=\pdv{f}{x}\odv{x}{t}+\pdv{f}{y}\odv{y}{t}+\pdv{f}{z}\odv{z}{t}
    \end{equation*}
\end{lem}
\begin{defn}(\textbf{Tangent plane})
    Given a differentiable function $f(x,y,z)$. Assume that the function passes through $P_{0}(x_{0},y_{0},z_{0})$ Let $\nabla f=f_{x}\mathbf{i}+f_{y}\mathbf{j}+f_{z}\mathbf{k}$. The tangent plane of $f$ at $P_{0}$ is given by
    \begin{equation*}
        f_{x}(x_{0},y_{0},z_{0})(x-x_{0})+f_{y}(x_{0},y_{0},z_{0})(y-y_{0})+f_{z}(x_{0},y_{0},z_{0})(z-z_{0})=0
    \end{equation*}
\end{defn}
\begin{defn}(\textbf{Linear approximation})
    Given a function $f$. The linear approximation of $f$ at $(x_{0},y_{0})$ is
    \begin{equation*}
        L(x,y)=f(x_{0},y_{0})+f_{x}(x_{0},y_{0})(x-x_{0})+f_{y}(x_{0},y_{0})(y-y_{0})
    \end{equation*}
\end{defn}
\begin{defn}(\textbf{Directional derivative})
    Given a unit direction $\mathbf{u}-u_{1}\mathbf{i}+u_{2}\mathbf{j}$ and a function $f(x,y)$. The directional derivative of $f$ in the direction of $\mathbf{u}$ at point $(x,y)$ is
    \begin{equation*}
        D_{\mathbf{u}}f(x,y)=\left.\odv{}{t}f(x+tu_{1},y+tu_{2})\right|_{t=0}
    \end{equation*}
\end{defn}
\begin{defn}(\textbf{Gradient vector})
    Given a differentiable function $f(x,y)$. The gradient vector of $f$ at $(x,y)$ is
    \begin{equation*}
        \nabla f(x,y)=\pdv*{f(x,y)}{x}\mathbf{i}+\pdv*{f(x,y)}{y}\mathbf{j}
    \end{equation*}
\end{defn}
\begin{thm}
    Given a differentiable function $f(x,y)$. The directional derivative of $f$ at $(x,y)$ in the unit direction $\mathbf{u}$ is given by
    \begin{equation*}
        D_{\mathbf{u}}f(x,y)=\nabla f(x,y)\cdot\mathbf{u}
    \end{equation*}
\end{thm}
\begin{thm}
    Given a differentiable function $f(x,y)$. Let $(a,b)$ be a point on the level curve $f(x,y)=c$. The gradient vector $\nabla f(a,b)$ is orthogonal to the level curve $f(x,y)=c$ at the point $(a,b)$.
\end{thm}
\begin{thm}
    Given a differentiable function $f(x,y)$. The equation of the tangent plane for the graph $z=f(x,y)$ at the point $(x_{0},y_{0},f(x_{0},y_{0}))$ is given by
    \begin{equation*}
        z=f(x_{0}+y_{0})+f_{x}(x_{0},y_{0})(x-x_{0})+f_{y}(x_{0},y_{0})(y-y_{0})
    \end{equation*}
\end{thm}
\begin{defn}(\textbf{Critical point})
    Given a differentiable function $f(x,y)$. A point $(a,b)$ is a critical point if the tangent plane at $(a,b)$ to the graph $z=f(x,y)$ is horizontal. This means that $f_{x}(a,b)=f_{y}(a,b)=0$ ($\nabla f(a,b)=\mathbf{0}$)
\end{defn}
\begin{thm}(Second derivative test)
    Let $f(x,y)$ be a differentiable function and $(x_{0},y_{0})$ be a critical point of $f$. Suppose that
    \begin{equation*}
        D(x,y)=\begin{vmatrix}
            f_{xx}(x,y) & f_{xy}(x,y)\\
            f_{yx}(x,y) & f_{yy}(x,y)
        \end{vmatrix}
    \end{equation*}
    If $D(a,b)>0$ and $f_{xx}(a,b)>0$, then $(a,b)$ is a local minimum.\\
    If $D(a,b)>0$ and $f_{xx}(a,b)<0$, then $(a,b)$ is a local maximum.\\
    If $D(a,b)<0$, then $(a,b)$ is a saddle point.\\
    Otherwise, it is inconclusive.
\end{thm}

\chapter{Multiple Integrations}
\begin{thm}(Fubini's Theorem for rectangular regions)
    Let $f(x,y)$ be a continuous function over a rectangle region $x\in[a,b]$ and $y\in[c,d]$. Then
    \begin{equation*}
        \int_{c}^{d}\int_{a}^{b}f(x,y)\,dx\,dy=\int_{a}^{b}\int_{c}^{d}f(x,y)\,dy\,dx
    \end{equation*}
\end{thm}
\begin{thm}(Fubini's Theorem for general regions)
    Let $R$ be a region on the $xy$-plane and $f(x,y)$ be a continuous function on $R$. Then
    \begin{equation*}
        \iint_{R}f(x,y)\,dx\,dy=\iint_{R}f(x,y)\,dy\,dx
    \end{equation*}
\end{thm}
\begin{thm}
    Let $f$ be a continuous function. We change the coordinate system from $(x,y,z)$ to $(u,v,w)$. We have
    \begin{equation*}
        \,dx\,dy\,dz=\abs{\frac{\partial(x,y,z)}{\partial(u,v,w)}}\,du\,dv\,dw
    \end{equation*}
    where $\abs{\frac{\partial(x,y,z)}{\partial(u,v,w)}}$ is the Jacobian determinant
    \begin{equation*}
        \abs{\frac{\partial(x,y,z)}{\partial(u,v,w)}}=\begin{vmatrix}
            \pdv{x}{u} & \pdv{x}{v} & \pdv{x}{w}\\
            \pdv{y}{u} & \pdv{y}{v} & \pdv{y}{w}\\
            \pdv{z}{u} & \pdv{z}{v} & \pdv{z}{w}
        \end{vmatrix}
    \end{equation*}
\end{thm}
\begin{thm}
    Let $x=r\cos\theta$ and $y=r\sin\theta$. Under polar coordinates $(r,\theta)$, we have
    \begin{equation*}
        \iint_{R}f(x,y)\,dA=\iint_{R}f(r\cos\theta,r\sin\theta)r\,dr\,d\theta
    \end{equation*}
\end{thm}
\begin{thm}
    Given a function $f(x,y)$. The surface area with equation $z=f(x,y)$ in region $D$ is given by
    \begin{equation*}
        \iint_{D}\sqrt{(f_{x}(x,y))^{2}+(f_{y}(x,y))^{2}+1}\,dA
    \end{equation*}
\end{thm}
\begin{thm}
    Let $x=r\cos\theta$, $y=r\sin\theta$. Under cylindrical coordinates $(r,\theta,z)$, we have
    \begin{equation*}
        \iiint_{D}f(x,y,z)\,dV=\iiint_{D}f(r\cos\theta,r\sin\theta,z)r\,dr\,d\theta\,dz
    \end{equation*}
\end{thm}
\begin{thm}
    Let $x=\rho\sin\phi\cos\theta$, $y=\rho\sin\phi\sin\theta$, $z=\rho\cos\phi$. Under spherical coordinates $(\rho,\theta,\phi)$, we have
    \begin{equation*}
        \iiint_{D}f(x,y,z)\,dV=\iiint_{D}f(\rho\sin\phi\cos\theta,\rho\sin\phi\sin\theta,\rho\cos\phi)\rho^{2}\sin\phi\,d\rho\,d\theta\,d\phi
    \end{equation*}
\end{thm}

\chapter{Vector calculus}
\begin{defn}(\textbf{Line integral of vector fields})
    Given a continuous vector field $\mathbf{F}(x,y,z)$ and a path $C$ which is parametrized by $\mathbf{r}(t)$ and $t\in[a,b]$. The line integral of $\mathbf{F}$ over $C$ is
    \begin{equation*}
        \int_{C}\mathbf{F}\cdot d\mathbf{r}=\int_{a}^{b}\mathbf{F}\cdot\mathbf{r}'(t)\,dt
    \end{equation*}
\end{defn}
\begin{defn}(\textbf{Conservative} vector field)
    A vector field $\mathbf{F}$ is conservative if and only if it is in the form of $\mathbf{F}=\nabla f$ where $f$ is a scalar function. The function $f$ is the \textbf{potential function} of the vector field $\mathbf{F}$.
\end{defn}
\begin{thm}
    Given a conservative vector field $\mathbf{F}=\nabla f$, where $f$ is a potential function. Along any path $C$ connecting from point $P_{0}(x_{0},y_{0},z_{0})$ to point $P_{1}(x_{1},y_{1},z_{1})$, the line integral is given by
    \begin{equation*}
        \int_{C}\mathbf{F}\cdot d\mathbf{r}=f(x_{1},y_{1},z_{1})-f(x_{0},y_{0},z_{0})
    \end{equation*}
\end{thm}
\begin{defn}(Closed path integral)
    Given a continuous vector field $\mathbf{F}$ and a path $C$ which is parametrized by $\mathbf{r}$. If $C$ is a closed path, the line integral of $\mathbf{F}$ over $C$ is
    \begin{equation*}
        \oint_{C}\mathbf{F}\cdot d\mathbf{r}
    \end{equation*}
\end{defn}
\begin{cor}
    For a conservative vector field $\mathbf{F}$, if $C_{1}$ and $C_{2}$ are two paths with the same initial and final positions, then
    \begin{equation*}
        \int_{C_{1}}\mathbf{F}\cdot d\mathbf{r}=\int_{C_{2}}\mathbf{F}\cdot d\mathbf{r}
    \end{equation*}
    Moreover, if $C$ is a closed path, then
    \begin{equation*}
        \oint_{C}\mathbf{F}\cdot d\mathbf{r}=0
    \end{equation*}
\end{cor}
\begin{defn}(\textbf{Curl})
    Given a vector field $\mathbf{F}=F_{x}\mathbf{i}+F_{y}\mathbf{j}+F_{z}\mathbf{k}$. The curl of the vector field $\mathbf{F}$ is given by
    \begin{align*}
        \nabla\times\mathbf{F}&=\left(\pdv*{}{x}\mathbf{i}+\pdv*{}{y}\mathbf{j}+\pdv*{}{z}\mathbf{k}\right)\times(F_{x}\mathbf{i}+F_{y}\mathbf{j}+F_{z}\mathbf{k})\\
        &=\begin{vmatrix}
            \mathbf{i} & \mathbf{j} & \mathbf{k}\\
            \pdv*{}{x} & \pdv*{}{y} & \pdv*{}{z}\\
            F_{x} & F_{y} & F_{z}
        \end{vmatrix}\\
        &=\left(\pdv{F_{z}}{y}-\pdv{F_{y}}{z}\right)\mathbf{i}+\left(\pdv{F_{x}}{z}-\pdv{F_{z}}{x}\right)\mathbf{j}+\left(\pdv{F_{y}}{x}-\pdv{F_{x}}{y}\right)\mathbf{k}
    \end{align*}
\end{defn}
\begin{defn}(\textbf{Simply-connected regions})
    A region $\Omega$ is simply-connected if $\Omega$ is connected and every closed loop in $\Omega$ can be contracted to a point continuously without leaving the region $\Omega$.
\end{defn}

\newpage
\begin{thm}(Curl test)
    Given a vector field $\mathbf{F}$ is defined and differentiable on a region $\Omega$.
    \begin{enumerate}
        \item If $\mathbf{F}=\nabla f$ for some scalar function $f$ defined on $\Omega$, then $\nabla\times\mathbf{F}=\mathbf{0}$ on $\Omega$.
        \item If $\nabla\times\mathbf{F}=\mathbf{0}$ and $\Omega$ is simply-connected, then $\mathbf{F}=\nabla f$ for some scalar function $f$ defined on $\Omega$.
    \end{enumerate}
\end{thm}
\begin{defn}(\textbf{Simple closed curves})
    A curve $C$ is a simple closed curve if the two endpoints coincide and it does not intersect itself at any point other than the endpoints.
\end{defn}
\begin{thm}(Green's Theorem)
    Let $C$ be a simple closed curve in $\mathbb{R}^{2}$ which is counter-clockwise oriented. Suppose the curve $C$ encloses region $R$. Let $\mathbf{F}(x,y)$ be a vector field which is defined and differentiable at every point in $R$. Then
    \begin{equation*}
        \oint_{C}\mathbf{F}\cdot d\mathbf{r}=\iint_{R}(\nabla\times\mathbf{F})\cdot\mathbf{k}\,dA
    \end{equation*}
\end{thm}
\begin{defn}(\textbf{Surface integrals})
    Given a surface $S$ parametrized by $\mathbf{r}(u,v)$ with $u\in[a,b]$ and $v\in[c,d]$, and a continuous, scaled-valued function $f(x,y,z)$. The surface integral of $f$ over the surface $S$ is
    \begin{equation*}
        \iint_{S}f\,dS=\int_{c}^{d}\int_{a}^{b}f(\mathbf{r}(u,v))\abs{\pdv{\mathbf{r}}{u}\times\pdv{\mathbf{r}}{v}}\,du\,dv
    \end{equation*}
\end{defn}
\begin{defn}(\textbf{Surface flux})
    Given a vector field $\mathbf{F}$ and a surface $S$. The surface flux of $\mathbf{F}$ through $S$ is
    \begin{equation*}
        \iint_{S}\mathbf{F}\cdot\mathbf{\hat{n}}\,dS
    \end{equation*}
    where $\mathbf{\hat{n}}$ is the unit normal vector to $S$ at each point.
\end{defn}
\begin{thm}
    Let $\mathbf{r}(u,v)$, with $u\in[a,b]$ and $v\in[c,d]$, be a parametrization of a surface $S$. The surface flux of a vector field $\mathbf{F}$ through $S$ can be computed by
    \begin{equation*}
        \iint_{S}\mathbf{F}\cdot\mathbf{\hat{n}}\,dS=\pm\int_{c}^{d}\int_{a}^{b}\mathbf{F}\cdot\left(\pdv{\mathbf{r}}{u}\times\pdv{\mathbf{r}}{v}\right)\,du\,dv
    \end{equation*}
    where the sign depends on the chosen convention of $\mathbf{\hat{n}}$.
\end{thm}
\begin{defn}(\textbf{Divergence})
    Given a differentiable vector field $\mathbf{F}=F_{x}\mathbf{i}+F_{y}\mathbf{j}+F_{z}\mathbf{k}$ in $\mathbb{R}^{3}$. The divergence of $\mathbf{F}$ is given by
    \begin{equation*}
        \nabla\cdot\mathbf{F}=\pdv{F_{x}}{x}+\pdv{F_{y}}{y}+\pdv{F_{z}}{z}
    \end{equation*}
\end{defn}
\begin{thm}
    Let $\mathbf{F}$ be a vector field. Then
    \begin{equation*}
        \nabla\cdot(\nabla\times F)=0
    \end{equation*}
    We can use this to detect whether a vector field is not a curl of another vector field.
\end{thm}
\begin{thm}(Stokes' Theorem)
    Let $S$ be an orientable, simply-connected surface in $\mathbb{R}^{3}$, and $C$ be the boundary curve of the surface $S$. Suppose $\mathbf{F}$ is a vector field which is defined and differentiable on the surface $S$, then
    \begin{equation*}
        \oint_{C}\mathbf{F}\cdot d\mathbf{r}=\iint_{S}(\nabla\times\mathbf{F})\cdot\mathbf{\hat{n}}\,dS
    \end{equation*}
    where $\mathbf{\hat{n}}$ is the unit normal vector to $S$, with direction determined by the right-hand rule.
\end{thm}
\end{document}